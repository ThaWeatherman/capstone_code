\section{Background}

This section is intended to serve as an overview of the techniques we will use and the threats
we work to detect.
While it is not a comprehensive review, it provides enough background knowledge to understand this
study.

\subsection{Machine Learning}

Machine learning is a field of Computer Science that is similar to pattern recognition.
It uses data to make predictions using models generated by algorithms.
It allows researchers or data scientists to make decisions based on structured or unstructured data.
The standard types of machine learning algorithms are supervised and unsupervised.
Somewhere in the middle is semi-supervised learning.

\subsubsection{Supervised Learning}

In supervised learning, the scientist trains the algorithm using training data.
The scientist creates a labeled data set consisting of some input data and a desired output value.
This data is fed to the algorithm so that the algorithm can properly classify similar data given
to it.
Then the scientist can supply other data sets of the same format to the algorithm for classification.

This form of learning requires knowledge of what different classifications look like before learning
begins.
This is satisfactory in many situations of classification, but it does not cover situations where
other classifications are unknown.
This setback is apparent in situations with security.
As an example, an intrusion detection system would ideally stop all possible threats rather than
only known threats.
A supervised algorithm would be unable to detect new and emerging threats.

Some examples of supervised learning algorithms are k-nearest neighbors, decisions trees, linear
regression, and support vector machines~\cite{scisuper}.
Each of these algorithms are powerful tools for supervised learning and are readily available
for use in standardized programming libraries.

\subsubsection{Unsupervised Learning}

In unsupervised learning, the scientist supplies unlabeled data to an algorithm.
No initial training set is used: the algorithm decides on how to classify or cluster the data on
its own.
Its intention is to handle known and unknown classifications or inputs and describing the data based
on classifications it decides on its own.
A main example of unsupervised learning is clustering.
Clustering attempts to cluster together similar looking data in an unlabeled data set.
Some of these algorithms require the scientist to provide the total number of desired clusters, while
others simply work on their own.
This is a useful technique for detecting unknown threats in security-related data.
However, its performance is not inherently better than supervised algorithms at classifying data.

\subsubsection{Semi-supervised Learning}

Semi-supervised learning is a cross between supervised and unsupervised learning.
It supplements unlabeled data with a small amount of labeled data to improve learning accuracy.
This could be introducing the labeled data into the whole set of unlabeled data and determining if
an algorithm correctly classifies or clusters the data based on the labeled data, or it could
introduce this known data into windows of the unlabeled data.
This form of learning is efficient and useful in many scenarios.
This is the type of learning we will utilize in our experiments.

\subsection{DNS Threats}

DNS is an application layer naming system for associating IP addresses with strings.
Everyone who uses the internet uses DNS\@.
It is hard to imagine that a simple mapping system could be used maliciously, but it is regularly
abused and misused.
Let us explore some of these threats.

\subsubsection{Cyber Squatting}

Cyber squatting boils down to profiting off of other peoples' trademarks.
Malicious users will register names that visually look similar to the names of legitimate websites.
They then either hope that users will mistype the legitimate names of websites, or they may spread
the link around the internet in hopes of driving traffic to their site.
They can profit off of these users using ads, they can serve malware, or they can phish the users.
This does not compromise DNS in any way, but it does maliciously use the service.

\subsubsection{Generated Names}

In recent years malware has grown significantly more sophisticated.
Some worms and botnets in particular have made a significant impact on the world.
Malware will algorithmically generate domain names that it will then contact for command and control
or for updates.
A very relevant example is the Conficker worm.
The Conficker worm's five variants used generated domain names to communicate daily with the worm
owner.
In the beginning, the worm would generate $250$ names each day and would attempt to contact each of
the names via HTTP until it made a successful connection and downloaded a new payload.
The fourth variant of Conficker generates $50,000$ names each day but only chooses $500$ of them
at random to contact~\cite{conficker}.
This allowed the worm to avoid attempts by registrars to curtail its spread.
The worm owner needed only to register one of the thousands of names, and the worms would eventually
find it for updates.
The Kraken and Torpig botnets also used generated names for similar purposes~\cite{algorithm}.
This clearly is an issue that needs to be solved.

\subsubsection{DNS Tunneling}

DNS tunneling is the act of hosts communicating using DNS packets.
DNS is not intended to be used as a communication layer, but it is not impossible to use it as one.
Tunneling is used to circumvent firewall rules.
A practical example is circumventing the paywall for internet usage on airplanes.
When airplane hotspots were first implemented, it was possible to circumvent the paywall by sending
all traffic through DNS.
The routers would block TCP traffic until payment was made for the service, but it paid no attention
to DNS requests and responses.
A more malicious example is data exfiltration.
After compromising a target, malware may need to exfiltrate sensitive data from the target.
Sending large amounts of traffic over various TCP ports may be easily detected, but DNS is inherently
trusted in many networks, and is not monitored as a result.
Hackers may also communicate with their malware using DNS tunneling rather than a normal TCP
connection.
Raman \etal{} proved that tunneling is a suitable method for malware delivery and data exfiltration
using a Metasploit module of their creation to deploy malware and communicate with it over DNS~\cite{tunnelpenetrate}.
They admitted their implementation is not as stealthy as it could be, but it is good enough to
circumvent most common network protections.
Tunneling is hard to address with simple network rules, and a good solution for detection is needed.
