\section{Future Work}

Unfortunately we were unable to use a large data set for this experiment.
Our data set was only $30 MB$ in size, which is significant enough for a proof of concept experiment,
but not large enough to simulate a real world use case.
Companies using a system like this have terabytes worth of network traffic to parse through.
We have been working with the Johns Hopkins University network administrator to gather a larger
data set.
She has collected $50 GB$ of purely DNS traffic and is in the process of running our extraction
scripts on it to both anonymize the data and extract out the features we need for this system.
Once this extraction process is completed, she will provide us with the data and we will run this
same experimentation process on the data.
We hope to reproduce similar results to our experiment, but on a larger scale.
This will work as a feasibility test for implementing a live system for finding threats in DNS data.

Upon completion of testing the large data set, we plan to build a live system using the process
from our experiment.
As we ran everything on an Apache Spark cluster, the system will be built using a Spark cluster.
Spark provides a live streaming API that can be used to process data in real time.
We would expand our experiment from working on batch jobs to working on streaming data.
This would be a product that a company could use for threat detection with minimal setup.

The advent of Big Data has opened new possibilities for threat detection.
Machine learning allows us to adapt to rising threats while handling known threats, and data
processing applications allow us to handle the large amounts of data flowing through company systems
each day.
The number of people connecting to the internet is increasing each day: it is critical that we create
systems which can adapt to new threats while also handling large loads of data.
This DNS threat model is but one example of a larger picture, but is a step towards better security for all.
