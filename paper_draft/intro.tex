\section{Introduction}

Every day there are thousands of newly registered domain names created.
Verisign reported that in the first quarter of 2015 there were $6,000,000$ newly registered domain names~\cite{verisign}.
That makes $66,667$ new names per day!
While some of those are for legitimate websites, it is unlikely that they all are.
As of writing today there have been more than $120,000$ new domain names registered, according to Domain Punch~\cite{domainpunch}.
A significant portion of these are used for malicious purposes.
I hardly see the purpose of the names $u356.com$ or $abf6.com$, but it is hard to argue that you
intend on using the name $f4ceb0ok.com$ for a legitimate purpose.

Hackers regularly use the domain name system (DNS) for nefarious purposes.
They squat on domain names that are similar to those of legitimate websites; they register thousands
of nonsensical names in a day; they send data in DNS payloads rather than using common transport
layer protocols for data, such as TCP and UDP.
All of this can be combatted, but not efficiently.
Black lists or white lists can be used to stop known bad names, but they are ineffective against
the thousands of newly registered domains that appear each day.

Machine learning is a field of study that is applicable throughout most of the Computer Science realm.
It can be applied to security, user experience, data science, and a host of other topics.
Machines can be taught to recognize patterns, and machine learning algorithms do just that.
The algorithms can be trained with example data, or they can be let loose on the target data with
no prior knowledge of the contents.
It is also possible to feed these unsupervised algorithms with small bits of example data to help
with clustering similar data together.
This is called semi-supervised learning.

In this paper, we propose that semi-supervised learning and be applied to detecting DNS threats.
We will specifically explore the ability of semi-supervised learning algorithms to detect malicious
domain names - such as those generated by some algorithm - as well as DNS tunneling.
We will show that this is an effective technique for detecting malicious DNS usage and that it could
be applied in actual security products.
