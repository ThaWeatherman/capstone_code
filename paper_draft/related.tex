\section{Related Work}

In this section we summarize prior work related to this project.
This includes work in semi-supervised learning as well as detecting DNS threats.

\subsection{Semi-supervised Learning}

Security is not a new application for machine learning.
Many researchers have attempted to use machine learning for intrusion detection, spam filtering,
or other topics, and have been successful.
However, many of these applications have relied on purely supervised or unsupervised learning techniques.
Semi-supervised learning has only gained popularity in the last few years, but has proven itself
a valuable tool, much more so than purely supervised or unsupervised methods.

A problem with machine learning is the tendency towards false alarms.
This is especially prevalent in unsupervised methods.
Intrusion detection is the most logical application of machine learning in the field of security.
A system is best when it can detect both known threats and unknown threats.
This requires unsupervised learning to detect the unknown threats, but can easily result in false alarms.
In Chiu \etal{}, they combatted false alarms in the Snort Intrusion Detection System (IDS), and were
able to reduce the false alarm rate by $85\%$ using semi-supervised learning~\cite{semialarm}.
In Chen \etal{}, the researchers proved that semi-supervised algorithms of their creation are able
to outperform both supervised and unsupervised methods~\cite{semidetect}.
It is also proven in Wang \etal{} that network traffic clustering is significantly improved using
semi-supervised learning~\cite{seminovel}.

\subsection{DNS Threat Detection}

There has been a significant amount of work in detecting DNS threats over the years, but very few
involve machine learning, if any.
The closest work in existence are non-learning oriented detection methods.

\subsubsection{Cyber Squatting}

Cyber squatting is a problem that is hard to combat.
It is not illegal to register domain names that are similar to the names of trademarked names.
It is also not easily detectable outside of using a black list.
This is tedious and ineffective against unknown malicious names.
In Chen \etal{}, the researchers were able to devise a browser extension that would warn users of
possible typing mistakes.
They named their system ATST, and found that it was able to detect up to $85\%$ of user errors.
It was also fast, delivering responses in as little as $300$ milliseconds~\cite{squat}.
Other systems for detection are built for website owners rather than end users.
Microsoft's Strider~\cite{strider}, Veralab's tool~\cite{veralab}, and dnstwist~\cite{elceef} are
all tools that a site owner can use to generate possible typo errors in their site names and check
if domains are registered under those names.\
dnstwist is the most recent and is quite effective.
However, none of these serve a purpose for detection on the user end.

\subsubsection{Generated Names}

Detecting generated names is not a new topic.
Two prior approaches involved using edit distance and machine learning.
In Yadav \etal{}, the researchers use edit distance to detect generated domain names and were
successful.
They also did a small experiment using linear regression (a supervised method) and were able to
successfully detect generated names.
Out of $3000$ transactions, their model detected $29$ as malicious.
Only $27$ were malicious and the other $2$ were false positives.
Unfortunately, they do not provide a metric of how many sites they might have missed with their
model~\cite{algorithm}.
In Antonakakis \etal{} the researches used both clustering and classification algorithms to
detect generated domain names.
From their data they were able to detect six known generation algorithms as well as six unknown
algorithms.
Their method proved machine learning can be successful in detection of this problem~\cite{usenix}.
However, they did not employ semi-supervised learning.

\subsubsection{Tunneling}

Most detection techniques for tunneling involve $n-grams$.
They use character analysis to examine the data in requests and responses of DNS traffic.
In Born \etal{}, they use character frequency analysis to detect tunneling.
They found that most domain and sub-domain names match character frequencies of the English language.
They then evaluated three tunneling tools --- Iodine, Dns2Tcp, and TCP-Over-DNS --- using frequency
analysis and were able to successfully detect the tools' attempts at tunneling.
The packets generated by each tool did not match English frequencies and could therefore be
categorized as tunnel packets~\cite{tunnels}.
It should be noted that if the tunnel packets used valid domain names and only used the data section
for actual data this methodology would not work.
The authors in ~\cite{bigram} used a similar technique and proved successful detection in real-time,
but again only checked the domain names in the DNS packets.
